% Created 2022-09-19 Mo 18:17
% Intended LaTeX compiler: pdflatex
\documentclass[11pt]{article}
\usepackage[utf8]{inputenc}
\usepackage[T1]{fontenc}
\usepackage{graphicx}
\usepackage{longtable}
\usepackage{wrapfig}
\usepackage{rotating}
\usepackage[normalem]{ulem}
\usepackage{amsmath}
\usepackage{amssymb}
\usepackage{capt-of}
\usepackage{hyperref}
\author{fakecrafter}
\date{\today}
\title{Organisation des Projektes}
\hypersetup{
 pdfauthor={fakecrafter},
 pdftitle={Organisation des Projektes},
 pdfkeywords={},
 pdfsubject={},
 pdfcreator={Emacs 28.2 (Org mode 9.5.5)}, 
 pdflang={English}}
\begin{document}

\maketitle
\tableofcontents


\section{Idee}
\label{sec:org173f495}
Unsere Idee war ein einfaches Snake-Spiel, sprich eine Schlange die sich auf einer 2D-Ebene in vier richtungen bewegen kann und versucht so viele Aepfel wie moeglich zu essen (mit dem Kopf dagegen kollidieren). Bei jedem Apfel wird die Schlange ein Feld laenger.
Wenn danach noch Zeit ist, gab es die Ueberlegung, ob man dieses Spiel nicht mit einem Pong-Spiel kombinieren koennte indem man die Schlaeger durch Schlangen ersetzt.
Ausserdem koennen noch weitere Features implementiert werden, wie zum Beispiel eine Traube, die die Laenge der Schlange verkuerzt, allerdings trotzdem Punkte hinzufuegt.
\section{Tools}
\label{sec:orgb1bf631}
Als Programmiersprache wird selbstverstaendlich Python benutzt und als externe Module werden \emph{random} und \emph{pygame} verwendet.
\section{Design}
\label{sec:orgdd1a3cb}
\subsection{Klasse Snake}
\label{sec:org21922b5}
\subsection{Klasse Item}
\label{sec:org8ed0010}
\subsubsection{Apple}
\label{sec:orgbaa6d28}
\subsubsection{Future Ideas}
\label{sec:org378a3c6}
\subsection{Spielablauf (main)}
\label{sec:org1b75167}
\section{Quellen}
\label{sec:org08704c9}
\begin{itemize}
\item \href{https://www.python-lernen.de/pygame-tutorial.htm}{pygame tutorial}
\end{itemize}
\section{Aufgaben}
\label{sec:orgefb77ce}
\begin{itemize}
\item Grundgeruest bauen (pygame)
\item Kaestchenstruktur entwickeln
\item Klasse fuer die Schlange
\begin{itemize}
\item draw()
\item movement()
\item appendBlock()
\item reset()
\item checkCollisionItem() -> Item
\item checkCollisionWall()
\item checkCollisionSelf()
\end{itemize}
\item Menu
\item Klasse fuer den Apfel
\end{itemize}
Note: Wer was macht, wird am 20.09 besprochen

\section{Momentan}
\label{sec:org53c693d}
Aktuell haben wir lediglich einige Dateien erstellt sowie ein github-repo. Ausserdem wurde sich auf Spiel und Struktur geeinigt
\section{Resultat}
\label{sec:org9786c06}
Der Kunde erhaelt ein vollstaendiges Spiel, das auf Windows-Betriebssystemen funktioniert. Ausserdem ist das Fenster skalierbar und kann an die Monitorgroesse angepasst werden
\section{Experten}
\label{sec:org4ebde35}
Ein Experte fuer pygame wird benoetigt. Er wird die draw()-Methoden implementieren sowie das Grundgeruest.
\end{document}